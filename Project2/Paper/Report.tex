\documentclass[12pt]{article}
\usepackage{amsmath}
\usepackage{graphicx}
\usepackage{float}
\usepackage[utf8]{inputenc}
\usepackage{cite}
\usepackage{setspace}
\usepackage{amsfonts}
\usepackage{mathrsfs}
\usepackage[margin=1.0in]{geometry}
\usepackage{enumitem}
\usepackage{listings}

\numberwithin{equation}{section}

\title{Programming a Linear Algebra Solution to the Problem of Two Electrons Trapped in a Spherical Harmonic Oscillator Well}
\author{Elizabeth Drueke}

\begin{document}
\maketitle

\begin{abstract}

\end{abstract}

\section{Introduction}
\label{sec:into}

\section{Theory}
\label{sec:theory}

Arguably the most important equation in quantum mechanics is Schrodinger's Equation, briefly stated as 

\begin{equation}
\label{eq:schrod}
H\Psi=E\Psi,
\end{equation}

\noindent where $H$ is the Hamiltonian for the system, $E$ is the energy of the system, and $\Psi$ is the state of the system.  Already, this equation is reminiscent of an eigenvalue problem, in which a matrix (the representation of $H$) acts upon a vector (the ket representation of the state) and returns that vector multiplied by some constant (an eigenvalue, the energy of the system).  Indeed, quantum mechanics is frequently formatted in this way, with state kets being represented by column vectors and the Hamiltonian by a Hermitian matrix \textbf{need citation - Griffiths? Sakurai?}  
\\\indent However, it is also quite reasonable to attempt to construct an analytical solution to the Schrodinger Equation, which is in fact a differential equation.  The derivatives arise from the quantum mechanical counterparts to classical quantities \textbf{more here - cite Sakurai}.  For example, quantum mechanical momentum is given by 

$$p\rightarrow -\frac{1}{i\hbar}\frac{\partial}{\partial x}$$

\noindent \textbf{check}  In the case of a particle confined to a well with a harmonic oscillator potential, we have a potential

\begin{equation}
\label{eq:potential}
V\left(r\right) = \[\begin{cases} \frac{1}{2}kr^{2} & \left|r\right| < a \\ 
\infty & \left|r\right| > a\end{cases}\]
\end{equation}

\noindent for some $a>0$ and where $m$ is the mass of the particle confined and $k=m\omega^{2}$ where $\omega$ is the harmonic oscillator frequency.  Thus, Schrodinger's Equation reads

\begin{equation}
\label{eq:schrodpart1}
here.
\end{equation}

\noindent Using separation of variables, and letting $\Psi\left(r,\theta,\phi\right)=R\left(r\right)\Theta\left(\theta,\phi\right)$, we can see that this is reduced to an angular equation

\begin{equation}
\label{eq:schrodang}
here
\end{equation}

\noindent and a radial equation

\begin{equation}
\label{eq:schrodrad}
-\frac{\hbar^{2}}{2m}\left(\frac{1}{r^{2}}\frac{d}{dr}\left(r^{2}\frac{d}{dr}\right)-\frac{l\left(l+1\right)}{r^{2}}\right)R\left(r\right)+V\left(r\right)R\left(r\right)=ER\left(r\right).
\end{equation}

\noindent In three dimensions, we have the solution

\begin{equation}
\label{eq:schrodsol}
E_{nl}=\hbar\omega\left(2n+l+\frac{3}{2}\right)
\end{equation}

\noindent for some $n,l\in\mathbb{N}\bigcup\{0\}$.  We can further simplify this by letting $R\left(r\right)=\frac{1}{r}u\left(r\right)$ to get

$$-\frac{\hbar}{2m}\frac{d^{2}}{dr^{2}}u\left(r\right)+\left(V\left(r\right)+\frac{l\left(l+1\right)}{r^{2}}\frac{\hbar^{2}}{2m}\right)u\left(r\right)=Eu\left(r\right).$$

\noindent We further restrict our solutions to be physically significant.  In particular, we require $u\left(0\right)=0$ and $u\left(\infty\right)=0$.  We can them make our equation dimensionless by introducing $\rho=\frac{1}{\alpha}r$ for some constant $\alpha$ of dimension length.  From there, we find

$$-\frac{\hbar^{2}}{2m\alpha^{2}}\frac{d}{d\pho^2}u\left(\rho\right)+\left(V\left(\rho\right)+\frac{l\left(l+1\right)}{\rho^{2}}\frac{\hbar^{2}}{2m}\right)=Eu\left(\rho\right)$$

\noindent In this project, we let $l=0$, and so, substituting our value for $V\left(\rho\right)$ from Eq.~\ref{eq:potential}, we see the Schrodinger equation becomes

$$-\frac{\hbar^{2}}{2m\alpha^{2}}\frac{d^{2}}{d\rho^{2}}u\left(\rho\right)+\frac{k}{2}\alpha^{2}\rho^{2}u\left(\rho\right)=Eu\left(\rho\right).$$

\indent Now, we have left $\alpha$ arbitrary up until this point, but now we choose it in such a way as to make the calculations simpler.  In particular, we let 

\begin{equation}
\label{eq:alpha}
\alpha=\left(\frac{\hbar^{2}}{mk}\right)^{1/4}
\end{equation}

\noindent We can see that $\alpha$ still has dimensions of length, as required.  In particular, we will notice that $\alpha$ defines the length scale of the problem (eg. $\alpha$ will be the Bohr radius for problems on the scale of one nucleus).  Then, letting

$$\lambda=\frac{2m\alpha^{2}}{\hbar^{2}}E,$$

\noindent we can rewrite Schrodinger's Equation as

\begin{equation}
\label{eq:schrodfinal}
-\frac{d^{2}}{d\rho^{2}}u\left(\rho\right)+\rho^{2}u\left(\rho\right)=\lambda u\left(\rho\right).
\end{equation}

\noindent\textbf{need citation - lecture notes}  
\\\indent It was shown in Project 1 \textbf{need citation - project 1 report} how we can set up a tridiagonal matrix in order to compute a numerical approximation to a second derivative.  In particular, we can define a $\rho_{min}=0$ and a $\rho_{max}$ and then, for some number of steps $n_{step}$, calculate a step size

$$h=\frac{\rho_{max}-\rho_{min}}{n_{step}}.$$

\noindent We then have $\rho_{i}=\rho_{min}+ih$ for $i=0,1,\ldots,n_{step}$, and we can write the Schrodinger equation as

\begin{equation}
\label{eq:schrodapprox}
-\frac{u\left(\rho_{i}+h\right)-2i\left(\rho_{i}\right)+u\left(\rho_{i}-h\right)}{h^{2}}+V_{i}=\lambda u\left(\rho_{i}\right)
\end{equation}

\noindent for $V_{i}=\rho_{i}^{2}$.  Thus, we have a tridiagonal matrix with diagonal elements

$$d_{i}=\frac{2}{h^{2}}+V_{i}$$

\noindent and off-diagonal elements

$$e_{i}=-\frac{1}{h^{2}}.$$

\noindent However, unlike in project 1, we are not solving a system of linear equations in this case, but rather an eigenvalue problem.  Solving this eigenvalue problem will be accomplished with the help of Jacobi's rotation algorithm \textbf{cite lecture notes}, which is outlined in Section~\ref{subsec:jacobi}.  
\\\indent \textbf{After we've solved eigenvalue problem how do we determine solution?  Various interacting/noninteracting cases?}

\subsection{The Jacobi Rotation Algorithm}
\label{subsec:jacobi}
\section{The Algorithm}
\label{sec:algorithm}

\subsection{Time Dependence}
\label{subsec:timedependence}

\section{Results and Benchmarks}
\label{sec:results}

\section{Conclusions}
\label{sec:conclusions}

\section{Bibliography}
\label{sec:bib}

\begin{enumerate}

\item e

\end{enumerate}

\end{document}