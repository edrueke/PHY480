\documentclass[12pt]{article}
\usepackage{amsmath}
\usepackage{graphicx}
\usepackage{float}
\usepackage[utf8]{inputenc}
\usepackage{cite}
\usepackage{setspace}
\usepackage{amsfonts}
\usepackage{mathrsfs}
\usepackage[margin=1.0in]{geometry}
\usepackage{enumitem}
\usepackage{listings}

\numberwithin{equation}{section}

\title{Developing a Numerical Solution to Ferromagnetic Materials}
\author{Elizabeth Drueke}

\begin{document}
\maketitle

\begin{abstract}
Modeling lattice-type structures under the effects of magnetization has been achieved with reasonable accuracy using the Ising model.  In order to accurately describe the larger of these systems, it is important to develop computer algorithms which can do large-scale computations in small amounts of time.  To this end, we develop a Monte Carlo process using the Metropolis Algorithm to study the effects of magnetization on two-dimensional lattice systems of various sizes and at various temperatures, and use the results of this analysis to determine the accuracy of our algorithm.  In partiular, we focus on the effect of various random number generators on the results, to ensure that we are not encountering a strong systematic bias from this source.
\end{abstract}

\section{Introduction}
\label{sec:intro}

To say that magnetism is an important physical phenomenon is a strong understatement.  Magnetism plays a strong role in day to day life on earth.  The earth's magnetic field is perhaps the most stunning example of this.  Caused by the electric fields resulting from the magma under the earth's crust \textbf{http://hyperphysics.phy-astr.gsu.edu/hbase/magnetic/magearth.html}, the earth's magnetic field has been used by explorers for centuries to navigate both on land and sea.  Today, magnets are used in some of the greatest experiments of our time, such as the Large Hadron Collider (LHC), which uses magnets both as a means of accelerating particles to high energies and to bend charged particles in detectors in an attempt to classify them.
\par In general, there are three kinds of magnetism: diamagnetism, paramagnetism, and ferromagnetism \textbf{http://web.hep.uiuc.edu/home/serrede/P435/Lecture\_Notes/Magnetism.pdf}, which is the subject of this paper.  The types of magnetism are distinguished by the behavior of the spins of atoms within the material being magnetized.  In ferromagnetic material, \textbf{What happens?} \textbf{cite Griffiths}.  The simplest theoretical model we have to describe ferromagnetism is the Ising model, described in Section\ref{sec:theory}.  
\par In this paper, we will discuss the Ising model and its importance, and then go into \textbf{need info and citations}.

\section{Theory}
\label{sec:theory}

\end{document}
